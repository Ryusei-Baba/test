% 2021 年度渡辺研究室
% 中間報告書テンプレート
% Version 2021-1.00

\RequirePackage{plautopatch}
\documentclass[12pt,a4paper,uplatex,dvipdfmx]{jsarticle}

\usepackage{amsmath, amssymb}
\usepackage{graphicx}
\usepackage{multirow}
\usepackage{float}
\usepackage{ascmac}
\usepackage{url}
\usepackage{lineno}
\usepackage{here}

\graphicspath{{fig/}} %% グラフィック用

%%%%%%%%%%%%%%%%%%%%%%%%%%%%%%%%%%%%%%%%%%%%%%%%%%%%%%%%%%%%%%%%%%%%%%%%%%%%%%%%%%%%
%%%% 寸法設定 %%%%

\setlength{\textwidth}{170truemm}		% 本文横幅
\setlength{\textheight}{230truemm}		% 本文縦幅
\setlength{\topmargin}{-5truemm}		% 上部マージン調整
\setlength{\oddsidemargin}{-4truemm}		% 奇数ページ左余白
\setlength{\evensidemargin}{\oddsidemargin}	% 偶数ページ左余白
\setlength\abovecaptionskip{-0.5truemm}		% 図キャプションと図との間隔
\setlength\belowcaptionskip{1.5truemm}		% 表キャプションと表との間隔

\begin{document}
東京工科大学メディア学部 渡辺研究室\\
\begin{center}
{\LARGE 卒業研究中間報告書} \\ \mbox{} \\
{\Large 感情相転移により生じる思念エネルギーの抽出法に関する研究} \\ \mbox{} \\
{\large m01yyxxx 三次元 萌子}
\end{center}

% ここから本文。
% 見出しや本文は各自で編集すること。
\section{概要}
\label{sec:introduction}
この文書は、LaTeX による中間報告書執筆の参考となる情報を記したものである。
LaTeX は「ラテフ」あるいは「ラテック」と呼称する。
(「ラテックス」という発音は明確に間違いとされている。)

\section{章立て}
\label{sec:section}
このスタイルでは、章を「\verb+\section+」、節を「\verb+\subsection+」で記載する
\footnote{卒論の場合、章は「\verb+\chapter+」、節は「\verb+\section+」となり、
「\verb+\subsection+」は項となる。}。
section や subsection コマンドの次に「\verb+\label{ラベル名}+」を付記しておくと、
文章中で章番号を参照できる。例えば、「概要は第\ref{sec:introduction}章」と記述できる。
章番号や節番号を数字で直接記述してしまうと、章や節の追加や削除があった場合に
ずれてしまい、修正が面倒となるので、必ず「\verb+\ref+」による参照を利用すること。

\section{段落と改行}
\label{sec:paragraph}

段落頭の字下げは自動で行われるため、全角スペースによる手動調整は不要であり、禁止である。
\LaTeX ソース中での改行は空行を挟まない場合は無視される。
ソース内では自分で編集しやすいように改行してよい。

このように、空行を挟むと改段落となる。
また、強制改行は\\このように \verb+\\+ で強制的に行うことができる。
しかし、この場合は段落の字下げもされないため、改段落を行う用途には空行を用いるべきで、
強制改行(\verb+\\+)は利用すべきではない。

しかしながら、例えば \verb+\verb+ 環境やインライン数式を用いる場合などで、
\verb+abcdefghijklmnopqrstuvwxyzABCDEFGHIJKLMNOPQRSTUVWXYZ+
というようにページ幅を超えてしまったり、前の行が間延びしてしまうようなケースがある。
そのような場合、\verb+\\+ を用いて強制改行により \\
\verb+abcdefghijklmnopqrstuvwxyzABCDEFGHIJKLMNOPQRSTUVWXYZ+
というように用いるとよい。

\section{箇条書き}
\label{sec:enum}

本章では、箇条書きの例を示す。

\subsection{数字付き箇条書き}
\label{subsec:enumerate}
\begin{enumerate}
 \item 数字の付いた箇条書きの例
 \item こんな感じで手順などを列挙
\end{enumerate}

数字を付けずに列挙したい場合は itemize 環境を使う。
このようにあるキーワードを指定して \verb+\begin(}+ と \verb+\end{}+ で
囲む範囲のことを○○環境と呼ぶ。

\subsection{数字無し箇条書き}
\label{subsec:itemize}

\begin{itemize}
 \item 順番などを伴わない箇条書きの例
 \item 材料や要素を純粋に列挙したい場合に使用
\end{itemize}

enumerate 環境や itemize 環境は、入れ子構造を持つことができる。
例えば enumerate 環境の場合、以下のようになる。
\begin{enumerate}
 \item 東京都
 \begin{enumerate}
  \item 八王子市
  \item 多摩市
 \end{enumerate}
 \item 神奈川県
 \begin{enumerate}
  \item 横浜市
  \item 川崎市
 \end{enumerate}
 \item 山梨県
\end{enumerate}

\section{図表と参照}
\label{sec:fig_tbl}

図を挿入する際は以下のように書く。
必ずキャプションを付けるとともに、図に対する説明を本文中で記載すること。
何かの手違いで図が表示されなくなったとしても、文章で意味が通じるくらいに説明するのを目安にすること。
以下の図 \ref{fig:sample} は、適当なサンプル画像である。

\begin{figure}[H]
  \centering
  \includegraphics[width=5.0truecm]{./fig/fig-sample.png}
  \caption{適当なサンプル}
% \url{http://www.this.is.sample.url/} % Web上のデータの場合、参照先URLを明記
  \label{fig:sample}
\end{figure}

様々な画像フォーマットを入力することが可能であるが、推奨するのは PNG 形式である。
EPS という形式に変換しておくと LaTeX のコンパイルが速いという利点があるが、
変換方法によっては画質が著しく劣化する場合があるので注意が必要である。
使用する画像ファイルは、このテンプレートのようにサブフォルダを作って分けておくことを推奨する。

図への参照は \verb+\label+ コマンドを用いて各図のキャプションにキーワードを付けておき、
文中で \verb+\ref+ コマンドによってキーワードを指定することで記述する。
キーワードは参照対象に応じてプリフィクスを付けることが望ましい。
以下の表 \ref{tbl:pre_list} に一般的に用いる参照対象ごとのプリフィクスを挙げる。

\begin{table}[H]
  \caption{ラベルに指定するキーワードのプリフィクス一覧}
  \label{tbl:pre_list}
  \centering
  \begin{tabular}{|l|l|r|} \hline
   参照対象	& プリフィクス \\ \hline
   章		& sec: \\ \hline
   節		& subsec: \\ \hline
   図		& fig: \\ \hline
   表 		& tbl: \\ \hline
   式   	& eqn: \\ \hline
  \end{tabular}
\end{table}

手作業でのナンバリングは非効率極まりない上に必ずミスが出るので行わないこと。

\section{数式}
\label{sec:eqn}

数式のインラインモードは \(x^2 + y^2 \leq 1\) のように表示させることができる.
インラインモードで「\verb+$...$+」を使うやり方は,
近年の LaTeX ではあまり推奨されていないが,その利用は妨げない.

ディスプレイ数式モードを利用する際に推奨するのは equation 環境である.
\begin{equation}
	\mathbf{A}_p = \frac{\mathbf{A}\cdot\mathbf{B}}{|\mathbf{B}|^2}\mathbf{B} .
	\label{eq:samp1}
\end{equation}
数式の参照は「\verb+\ref+」ではなく「\verb+\eqref+」を用いる.
上記の数式を参照すると「式\eqref{eq:samp1}」となる.
このように,\verb+\eqref+ を用いた場合は数式中と同じ様式の括弧がつく.

また,複数行にわたる数式を表示したい場合は align 環境を用いることを推奨する.
以下の式\eqref{eq:samp2}にその例を示す.

\begin{align}
	& \begin{bmatrix}
	a_{11} & a_{12} & \cdots & a_{1n} \\
	a_{21} & a_{22} & \cdots & a_{2n} \\
	\vdots & \vdots & \ddots & \vdots \\
	a_{m1} & a_{m2} & \cdots & a_{mn} \\
	\end{bmatrix}
	\otimes
	\begin{bmatrix}
	b_{11} & b_{12} & \cdots & b_{1n} \\
	b_{21} & b_{22} & \cdots & b_{2n} \\
	\vdots & \vdots & \ddots & \vdots \\
	b_{m1} & b_{m2} & \cdots & b_{mn} \\
	\end{bmatrix} \notag \\
	& \qquad \qquad = \sum_{i}^{m}\sum_{j}^{n}a_{ij}b_{ij} .
	\label{eq:samp2}
\end{align}

なお、LaTeX 関係の書籍や Web による情報で、
複数行の数式に eqnarray 環境を紹介している文献があるが、
eqnarray 環境は最近の LaTeX では幾つかのパッケージと同時に利用すると
問題が発生することがあるため、利用は推奨しない.

具体的な数式の記述方法については解説書籍や Web の情報を参照してほしい。

\section{参考文献}
\label{sec:bib}

参考文献リストの作成は、BibTeX を用いることを推奨する。
このテンプレートでは「Bibtex.bib」というファイルで文献リストを記述してある。

文献の参照は、リスト上で文献に付けたキーワードをciteコマンドによって指定することで記述する。
ここでは、日本語論文\cite{nito}、英文論文\cite{nowrouzezahrai}、
日本語書籍\cite{godzilla}、英語書籍\cite{gpu-gems2}、
学部卒業論文\cite{takano}\footnote{学部卒業論文なのに bib ファイル中のカテゴリで修士論文用の
「masterthesis」を使っているのは、学部卒業論文というカテゴリが
BibTeX には存在していないからである。
国際的には、(日本のように)学部生に卒業論文を書かせる国はあまり多くはない。}
、修士論文\cite{abeM}、博士論文\cite{takeuchi3}、
URL\cite{zbrush} の各例を示す。特に、URL の参照では参照年日時を記載するのを忘れないこと。

文献を1つも参照していない状態で LaTeX のコンパイルをかけるとエラーとなるので注意すること。
生成したPDFファイル中で参照がうまくできていない場合には参照番号ではなく?記号が表示される。

BibTeX のリスト作成方法については、同梱してある Bibtex.bib を参考にすること。
文献の属性の種類や、設定するべきステータスについてもWebを参照すること。
論文データベースサイトでは BibTeX の記述形式によるテキストを出力してくれるところもあるので、
利用できると便利である。
リストの記述順は一切気にする必要がなく、
参考文献に挙げないものが含まれていても問題無いので、関連しそうな文献は全てリスト化しておくとよい。

BibTeX 中の項目によっては、単語の大文字が自動的に小文字に変換されることがある。
しかし、それが望ましくない場合(例えば「3DCG」を「3dcg」に変換されてしまう場合など)は、
変換したくない単語を波括弧(\verb+{}+)で囲うことで回避できる。
ただし、項目全体(例えば論文タイトルの最初から最後まで全部)を波括弧で囲ってしまうと、
LaTeX の整形が崩れてしまうことがあるので、波括弧は単語単位で囲うべきである。

\section{中間発表の構成}
\label{sec:structure}
中間報告書は、以下のような章立てが推奨される。

\begin{enumerate}
 \item 概要
 \item 研究方針
 \item 現状報告
 \item 今後の予定
 \item 参考文献
\end{enumerate}

「概要」は、主に研究の背景と目的について解説を行う章である。
背景では、この研究の目的を理解する上で重要となる情報の解説があれば、
それについても述べておく。

「研究方針」は、目的を達成するための手法について詳細に解説を行う章である。
実験が主体となる場合は、その実験内容について詳しく述べる。

「現状報告」は、現時点で出来ていることについて記載する。
実行画面や予備実験結果などがここに当てはまる。

「今後の予定」は、中間発表以降での研究計画を述べる。
ガントチャートなどを利用した解説でもよい。

\bibliography{Bibtex}
\bibliographystyle{junsrt}

\end{document}
